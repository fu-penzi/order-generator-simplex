\documentclass[a4paper,12pt,twoside]{article}

\usepackage[utf8]{inputenc}
\usepackage{polski}
\usepackage{verbatim}
\usepackage{lmodern}% http://ctan.org/pkg/lm

\title{SPECYFIKACJA FUNKCJONALNA\\Generator zamówień}
\date{7 listopada 2020}
\author{Piotr Nowak}

\begin{document}
\maketitle

\section{Opis ogólny}
\subsection{Opis}
\noindent{\bf Generator zamówień\\}
Program  na podstawie pliku tekstowego zawierającego informacje o producentach szczepionek, aptekach i połączeniach między nimi generuje listę zamówień tak, aby zapotrzebowanie aptek na szczepionki zostało zaspokojone a  całkowite opłaty były jak najmniejsze.
\subsection{Język}
Projekt został wykonany w języku programowania C.
\section{Dane wejściowe i wyjściowe}
\subsection{Argumenty wejściowe}
Program wywołujemy pisząc:

\medskip

   \quad \texttt{./orders [opcja]...} 

\bigskip

gdzie dopuszczalne opcje to:
\begin{description}

\item[\texttt {-i }] Dane wejściowe z pliku tekstowego 

\item[\texttt {-o }] Wyjściowy plik tekstowy zawierający listę zamówień

Przykładowe wywołanie:
\medskip
\begin{verbatim}
		./orders -i przykład_danych.txt -o zamowienia.txt 
\end{verbatim}
\end{description}
\newpage
\subsection{Format plików wejściowych}
Plik tekstowy w formie:

\begin{verbatim}
# Producenci szczepionek (id | nazwa | dzienna produkcja)
0 | BioTech 2.0 | 900
1 | Eko Polska 2020 | 1300
2 | Post-Covid Sp. z o.o. | 1100
...
# Apteki (id | nazwa | dzienne zapotrzebowanie)
0 | CentMedEko Centrala | 450
1 | CentMedEko 24h | 690
2 | CentMedEko Nowogrodzka | 1200
...
# Połączenia producentów i aptek (id producenta | id apteki 
| dzienna maksymalna liczba dostarczanych szczepionek | 
koszt szczepionki [zł])
0 | 0 | 800 | 70.5
0 | 1 | 600 | 70
0 | 2 | 750 | 90.99
1 | 0 | 900 | 100
1 | 1 | 600 | 80
1 | 2 | 450 | 70
2 | 0 | 900 | 80
2 | 1 | 900 | 90
2 | 2 | 300 | 100
...
\end{verbatim}
\subsection{Format wyników}
Plik tekstowy przedstawiający zamówienia i ich koszt w formie:
\begin{verbatim}
producent -> apteka [Koszt = liczba szczepionek * cena
za szczepionkę = ...zł]
...
Opłaty całkowite = suma kosztów wszystkich zamówień[zł]
\end{verbatim}
\begin{verbatim}
np.
BioTech 2.0     -> CentMedEko Centrala [Koszt = 300 * 70.5 = 21150 zł]
Eko Polska 2020 -> CentMedEko Centrala [Koszt = 150 * 100 = 15000 zł]
/*
...
kolejne wiersze opisujące ustalone połączenia pomiędzy
 producentami a aptekami
...
*/
Opłaty całkowite: 36150 zł 
\end{verbatim}
\section{Scenariusz działania programu}
\begin{enumerate}
\item Program otwiera plik tekstowy, którego nazwa została podana na wejściu, i pobiera z niego informacje.
\item Algorytm generuje zamówienia na podstawie danych z pliku wejściowego.
\item Zapisuje wygenerowane zamówienia do pliku tekstowego.
\item Program kończy działanie.
\end{enumerate}
\section{Sytuacje wyjątkowe}
\begin{enumerate}
\item Niewłaściwe wywołanie programu(np. niewłaściwa flaga).
\par Wyświetlenie instrukcji wywołania programu:
\begin{verbatim}
Użycie:
-i  -  plik wejściowy
-o  -  plik wyjściowy
\end{verbatim}
\item Błąd podczas otwierania pliku wejściowego
\par Program wyświetla komunikat: "Wystąpił błąd przy czytaniu pliku [Nazwa pliku]." i kończy działanie.
\item Plik wejściowy niezgodny z ustaloną konwencją
\par Program wyświetla informacje o miejscu, w którym występują niezgodności.
\end{enumerate}
\end{document}

